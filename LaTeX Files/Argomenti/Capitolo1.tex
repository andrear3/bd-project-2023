\chapter{Requisiti identificati}

\section{Richiesta del lavoro assegnato}
Il progetto ha come obiettivo la realizzazione di una base di dati in grado di gestire un sistema basato sul raccoglimento di foto geolocalizzate inserite dagli utenti, i quali hanno anche l'opzione di condividerle. 

\section{Funzionalità principali}
Il sistema è in grado di tracciare ogni foto secondo specifiche caratteristiche tra cui: utente, codice identificativo, dispositivo con cui è stata scattata, luogo, data della fotografia e (se specificato) soggetto.
Una foto può contenere più soggetti, che però sono identificati in modo unico nel sistema.
L'utente ha la possibilità di osservare la sua galleria personale, che conterrà le sue foto personali, ma può anche partecipare a collezioni condivise con altri utenti, che possono contenere le foto dei partecipanti ad esse.
Il sistema può anche gestire la visibilità di ogni foto, andando a specificare se una foto deve essere privata (rendendola invisibile ad utenti esterni) o se deve essere pubblica (rendendola visibile all'interno delle collezioni).
\break
In aggiunta:
\break \\
 - E' possibile eliminare una foto, rendendola indisponibile all'utente originale, ma ancora accessibile se era già presente in una collezione pubblica al momento della cancellazione.
\break \\
- E' possibile eliminare un utente dal sistema, rimuovendo così tutte le sue foto, fatta eccezione per quelle che contengono come tag altri utenti.
\break \\
- E' possibile creare un video con le foto nel sistema, dotato di codice unico e caratteristiche come durata e descrizione.
\break \\
Il sistema, inoltre, può svolgere alcune operazioni di recupero informazioni, come il recupero delle foto per soggetto, per luogo e la top 3 dei luoghi più fotografati.

\section{Funzionalità aggiuntive}
\vspace{1em}
- E' possibile anche accedere ad un'apposita galleria personale dei video.
\break \\
- La funzionalità di rimozione utente andrà anche a rimuovere i video dal sistema.
\break \\
- Ogni Collezione Condivisa sarà caratterizzata da un proprio nome, e conterrà anche la data in cui un utente ne è entrato a far parte.
\break \\