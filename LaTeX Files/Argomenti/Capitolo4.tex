\chapter{Modello Logico}


\section{Schema Logico}
\vspace{2em}

\textbf{PHOTO}(\underline{Photo\textunderscore Code},Device,Scope,Photo\textunderscore Date)\\

\bigskip

\textbf{TAG}(\underline{Tag\textunderscore Name})\\

\bigskip

\textbf{USER}(\underline{Nickname},Name,Surname,BirthDate,Gender)\\

\bigskip

\textbf{LOCATION}(\underline{Location\textunderscore Name},X\textunderscore Coordinates,Y\textunderscore Coordinates,Photo\textunderscore Count)\\

\bigskip

\textbf{VIDEO}(\underline{Video\textunderscore Code},Video\textunderscore Lenght,Video\textunderscore Desc,Video\textunderscore Title)\\

\bigskip

\textbf{PUBLIC\textunderscore COLLECTION}(\underline{Collection\textunderscore Name })\\

\bigskip

\textbf{IS\textunderscore IN\textunderscore VIDEO}(\underline{Video\textunderscore Code,Photo\textunderscore Code})\\
IS\textunderscore IN\textunderscore VIDEO.Video\textunderscore Code \leftarrow VIDEO.Video\textunderscore Code\\
IS\textunderscore IN\textunderscore VIDEO.Photo\textunderscore Code \leftarrow PHOTO.Photo\textunderscore Code\\

\bigskip

\textbf{PARTECIPATING\textunderscore USERS}(\underline{Join\textunderscore Data, Collection\textunderscore Name, Nickname})\\
PARTECIPATING\textunderscore USERS.Collection\textunderscore Name \leftarrow PUBLIC\textunderscore COLLECTION.Collection\textunderscore Name\\
PARTECIPATING\textunderscore USERS.Nickname \leftarrow USER.Nickname\\

\bigskip

\textbf{VIDEO\textunderscore MADE\textunderscore BY}(\underline{Video\textunderscore Code,Nickname})\\
VIDEO \textunderscore MADE\textunderscore BY.Video\textunderscore Code\leftarrow VIDEO.Video \textunderscore Code\\
VIDEO\textunderscore MADE\textunderscore BY.Nickname \leftarrow USER.Nickname\\
\bigskip
\textbf{USER\textunderscore TAG}(\underline{Nickname,Photo\textunderscore Code})\\
USER\textunderscore TAG.Nickname \leftarrow USER.Nickname\\
USER\textunderscore TAG.Photo\textunderscore Code \leftarrow PHOTO.Photo\textunderscore Code\\
\bigskip
\pagebreak
\textbf{PHOTO\textunderscore TAG}(\underline{Tag\textunderscore Name,Photo\textunderscore Code})\\
PHOTO\textunderscore TAG.Tag\textunderscore Name \leftarrow TAG.Tag\textunderscore Name\\
PHOTO\textunderscore TAG.Photo\textunderscore Code \leftarrow PHOTO.Photo\textunderscore Code\\

\bigskip
\textbf{PHOTO\textunderscore TAKEN\textunderscore IN}(\underline{Location\textunderscore Name,Photo\textunderscore Code})\\
PHOTO\textunderscore TAKEN\textunderscore IN.Location\textunderscore Name \leftarrow LOCATION.Location\textunderscore Name \\
PHOTO\textunderscore TAKEN\textunderscore IN.Photo\textunderscore Code \leftarrow PHOTO.Photo\textunderscore Code\\
\bigskip
\textbf{PHOTO\textunderscore MADE\textunderscore BY}(\underline{Nickname,Photo\textunderscore Code})\\
PHOTO\textunderscore MADE\textunderscore BY.Nickname \leftarrow USER.Nickname \\
PHOTO\textunderscore MADE\textunderscore BY.Photo\textunderscore Code \leftarrow PHOTO.Photo\textunderscore Code\\
\bigskip
\textbf{SHARED\textunderscore PHOTO}(\underline{Photo\textunderscore Code,Collection\textunderscore Name})\\
SHARED\textunderscore PHOTO.Photo\textunderscore Code \leftarrow  PHOTO.Photo\textunderscore Code\\
SHARED\textunderscore PHOTO.Collection\textunderscore Name \leftarrow PUBLIC\textunderscore COLLECTION.Collection\textunderscore Name \\

\pagebreak
\section{Funzioni, Procedure e Trigger Functions}

\vspace{2em}

\subsection{Funzioni e Procedure}

\vspace{2em}

\textbf{rendi\textunderscore foto\textunderscore privata}
\break
Procedure che aggiorna la visibilità di una foto a 'Private'.
\break

\textbf{rendi\textunderscore foto\textunderscore pubblica}
\break
Procedure che aggiorna la visibilità di una foto a 'Public'.
\break

\textbf{elimina\textunderscore foto}
\break
Procedure che permette l'eliminazione di una foto.
\break

\textbf{galleriapersonale()}
\break
Funzione che permette all'utente di visualizzare le proprie foto.
\break

\textbf{galleriapersonalevideo()}
\break
Funzione che permette all'utente di visualizzare i propri video.
\break

\textbf{foto\textunderscore stesso\textunderscore soggetto()}
\break
Funzione che permette di visualizzare tutte le foto che hanno in comune lo stesso soggetto.
\break

\textbf{foto\textunderscore stesso\textunderscore luogo()}
\break
Funzione che recupera tutte le foto che condividono lo stesso luogo.
\break

\textbf{top\textunderscore 3\textunderscore luoghi()}
\break
Funzione che permette di visualizzare le 3 località più immortalate.
\break

\textbf{video\textunderscore foto()}
\break
Funzione che permette di visualizzare tutte le singole foto che compongono un determinato video.
\break

\pagebreak

\subsection{Trigger Functions}

\vspace{2em}
\textbf{Delete\textunderscore Photo()}
\break
Trigger function che imposta lo scope a 'Eliminated' delle fotografie che sono rimosse dal database.
\break

\textbf{Delete\textunderscore User()}
\break
Trigger function che gestisce i dati di un utente eliminato dal database.
\break

\textbf{Location\textunderscore Count()}
\break
Trigger function che viene attivata ad ogni inserimento di una nuova foto incrementando il counter della location associata.
\break

\textbf{Location\textunderscore Count\textunderscore Subtract()}
\break
Trigger function che viene attivata ad ogni rimozione di una foto diminuendo il counter della location associata.
\break

\textbf{NewUser\textunderscore Collection()}
\break
Trigger function che viene attivata nel momento in cui viene inserita una foto in una galleria condivisa aggiungendo l'utente tra la lista dei partecipanti se non vi è presente.
\break

\textbf{Private\textunderscore Photo()}
\break
Trigger function che viene attivata a seguito di una modifica di una foto rendendo lo scope 'Private' ed eliminando la foto dalle gallerie condivise.
\break

\textbf{Public\textunderscore Photo()}
\break
Trigger function che viene attivata a seguito della condivisione di una foto modificando lo scope da 'Private' a 'Public'.
\break



